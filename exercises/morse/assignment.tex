\documentclass[a4paper,11pt]{article}

\input ../include/preamble.tex

% SECTIONS
%
% * Introduction
% * Morse codes
% * The decoder
% * The encoder
% * The Morse codes


\begin{document}

% ================================================== %
% == Title == %
% ================================================== %

\title{
    \textbf{Morse Encoding}\\
    \large{Programming II - Elixir Version}
}
\author{Johan Montelius}
\date{Spring Term 2024}
\maketitle
\defaultpagestyle


% ================================================== %
% == Introduction == %
% ================================================== %

\section*{Introduction}

Morse codes were used in the days of telegraphs and manual radio
communication. It is similar in the idea to Huffman coding in that it
uses fewer symbols for frequent occurring characters and more symbols
for the infrequent ones. You task is to write a decoder for Morse
signals and decode two secret messages.   


% ================================================== %
% == Morse codes == %
% ================================================== %

\section{Morse codes}

There are several standards for Morse codes and we will here use a
slightly extended version since we also want to code some special
character. The Morse code uses, as you probably know, long and short
(often pronounced {\em di} and {\em da}) to encode characters. You
might therefore think that is identical to Huffman codes but there is
a difference. In Morse coding we have a special signal that tells us
when one character ends and the next start. The pause between
characters is necessary in order to decode a message.

The code for 'a' is {\em di-da}, 'i' is {\em di-di} and 'l' is {\em di-da-di-di}.  If
we just had the sequence {\em di-da-di-di} we would not know if this
was ``ai'' or ``j''; we need a third signal, the pause, to tell the
difference. A sequence {di-da-pause-di-di-pause} is then decoded as ``ai''.

How does this change the structure of our decoding tree? In a Huffman
tree we only have characters in the leafs and when we hit a leaf we
know that we have a complete character and can start from the root
again. In a Morse tree we can finish anywhere along the path to a
leaf. We thus have characters in every node of the tree (apart from the
root).


% ================================================== %
% == The decoder == %
% ================================================== %

\section{The decoder}

The Morse codes that we will use are given in the decoding tree in
the Appendix. As you see we have represented the table as a tree were
each node is on the form:

\begin{minted}{elixir}
{:node, character, long, short}
\end{minted}

An empty tree is represented by the value {\tt :nil} and the root holds
dummy value instead of a character; if everything works fine we
will never have to see the nil nor the dummy value.

To decode a message you only have to choose the {\em long branch} when you
hear a long signal and a {\em short branch} when you hear a short
signal. When you hear the pause you have decoded a character and can
start from the root again. 

Implement a function {\tt decode(signal, tree)} that takes a {\em
  signal} and the decoding tree and returns a decoded message. The
signal is in the form of a string with dots and dashes {\tt '.-
  . .-.. .-.. ---'} (i.e. a charlist with ASCII characters $45$, $46$
and $32$ (dash, dot and space), don't use the integers in yoru code
but the elixir way of writing ASCII codes i.e. {\tt ?.}, {\tt ?-} and
{\tt ?\\s})

Decode the secret messages below. If you cut and paste the code, make
sure that you don't have carriage-return etc in the string. The string
should only contain the dash, dot and space characters. 

\begin{verbatim}
  '.- .-.. .-.. ..-- -.-- --- ..- .-. ..-- 
    -... .- ... . ..-- .- .-. . ..-- 
    -... . .-.. --- -. --. ..-- - --- ..-- ..- ...'
\end{verbatim}

\begin{verbatim}
  '.... - - .--. ... ---... .----- .----- .-- .-- 
    .-- .-.-.- -.-- --- ..- - ..- -... . .-.-.- -.-. 
    --- -- .----- .-- .- - -.-. .... ..--.. ...- .----. 
    -.. .--.-- ..... .---- .-- ....- .-- ----. .--.-- 
    ..... --... --. .--.-- ..... ---.. -.-. .--.-- 
    ..... .----'
\end{verbatim}

If you by accident have two spaces between to characters this might be
decoded as {\tt :na} which then will generate a string that is not a
proper string. To solve this problem you might add a rule for the case
where you're in the root of the tree (the character is {\tt :na}) and
you hear a space. Then you simply do nothing and just start from the
root again. The rule can be made general so that any unknown character
will simply make the decoder start from the root.


% ================================================== %
% == The encoder == %
% ================================================== %

\section{The encoder}

To encode messages we simply need the table that gives us codes for
each character. You could of course use the tree directly to find the
code of a character but this requires searching through the whole tree
for each caracter. Why not traverse the tree once and build a map structure
that maps characters to codes. The codes are best represented as
charlist. 

First implement a fucntion that takes the morse tree and returns an
encoding table as a map. Is then a simple task to implement a function
that takes a message (encoded as a charlist) and encodes each
character of the message. When you construct the encoded message make
sure that you include a pause between each code sequence. The message
'sos' is thus encoded {\tt '... --- ... '} with spaces between the
sequences.


\pagebreak


% ================================================== %
% == The Morse codes  == %
% ================================================== %

\section{The Morse codes}

\begin{minted}{elixir}
def tree do
  {:node, :na,
    {:node, 116,
      {:node, 109,
        {:node, 111,
          {:node, :na, {:node, 48, nil, nil}, {:node, 57, nil, nil}},
          {:node, :na, nil, {:node, 56, nil, {:node, 58, nil, nil}}}},
        {:node, 103,
          {:node, 113, nil, nil},
          {:node, 122,
            {:node, :na, {:node, 44, nil, nil}, nil},
            {:node, 55, nil, nil}}}},
      {:node, 110,
        {:node, 107, {:node, 121, nil, nil}, {:node, 99, nil, nil}},
        {:node, 100,
          {:node, 120, nil, nil},
          {:node, 98, nil, {:node, 54, {:node, 45, nil, nil}, nil}}}}},
    {:node, 101,
      {:node, 97,
        {:node, 119,
          {:node, 106,
            {:node, 49, {:node, 47, nil, nil}, {:node, 61, nil, nil}},
            nil},
          {:node, 112,
            {:node, :na, {:node, 37, nil, nil}, {:node, 64, nil, nil}},
            nil}},
        {:node, 114,
          {:node, :na, nil, {:node, :na, {:node, 46, nil, nil}, nil}},
          {:node, 108, nil, nil}}},
      {:node, 105,
        {:node, 117,
          {:node, 32,
            {:node, 50, nil, nil},
            {:node, :na, nil, {:node, 63, nil, nil}}},
          {:node, 102, nil, nil}},
        {:node, 115,
          {:node, 118, {:node, 51, nil, nil}, nil},
          {:node, 104, {:node, 52, nil, nil}, {:node, 53, nil, nil}}}}}}
end
\end{minted}

\end{document}
